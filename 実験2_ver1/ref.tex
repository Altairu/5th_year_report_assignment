% !TEX root = main.tex
\newpage
%%%%%%%%%%%%%%%%%%%%%%%%%%%%%%%%%%%%%%%%%%%%%%%%%%%%%%%%%%%%%%%%%%%%%%%%
\begin{center}
	\section*{参\,考\,文\,献}                      %% ここに番号をつけない
\end{center}
\addcontentsline{toc}{section}{参考文献} %% 目次に番号をつけない
%%%%%%%%%%%%%%%%%%%%%%%%%%%%%%%%%%%%%%%%%%%%%%%%%%%%%%%%%%%%%%%%%%%%%%%%

\begin{thebibliography}{10}
	\bibitem{ref1} 川田(編著), 東, 市原, 浦久保, 大塚, 甲斐, 國松, 澤田, 永原, 南 : 倒立振子で学ぶ制御工学, 森北出版 (2017)
	\bibitem{ref2} 川谷, 外川 : 現代制御理論を使った倒立振子の実験 [1]–[3], トランジスタ技術, No. 5, pp. 315–322, No. 6, pp. 367–373, No. 7, pp. 363–370 (1993)
	\bibitem{ref3} 「初学者のための図解でわかる制御工学」特集号(基礎編), システム/制御/情報, Vol. 56, No. 4 (2012)
	\bibitem{ref4} 大山 : たしなみながら学ぶセンサ制御 -フィードバック制御から現代制御理論の応用まで, インターフェース, No. 9, pp. 80–119 (1993)
	\bibitem{ref5} 大山, 工藤, 岡本, 藤沢 : 現代制御理論に基づいたディジタル制御系の設計法, インターフェース, No. 12, pp. 215–250 (1986)
	\bibitem{ref6} 川田 : MATLAB/Simulink による制御工学入門, 森北出版 (2020)
	\bibitem{ref7} 川田 : MATLAB/Simulink による現代制御入門, 森北出版 (2011)
	\bibitem{ref8} 川田 : MATLAB/Simulink と実験で学ぶ制御工学 -PID 制御から現代制御まで-, TechShare (2013)
	\bibitem{ref9} RSコンポーネンツ 「ポテンショメーター完全ガイド~多彩な使い道と種類を徹底解説」\\
	〈https://jp.rs-online.com/web/content/discovery/ideas-and-advice/potentiometers-guide?srsltid=AfmBOopoFLKTmQ70MlD\_a9sccNLFYl4ibGXCN-zv51aO\_NanDwfYN5FM〉(2024 年 10 月 27 日取得)
	\bibitem{ref10} CONTEC 「アナログ入出力の基礎知識・用語解説」\\
	〈https://www.contec.com/jp/support/basic-knowledge/daq-control/analog-io/〉(2024 年 10 月 27 日取得)
\end{thebibliography}

% *******************************************