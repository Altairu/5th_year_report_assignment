% !TEX root = main.tex

\section{実験2-1:画像処理実験}

\subsection{実験概要}
本実験では,手先に装着したRGB-Dカメラから取得した画像を処理し,
対象物体の検出および3次元位置情報の取得を実験する.
マニピュレータの卓上には,右角度の積み木(赤色,青色,黄色)および収納台が設置されており,
実験では積み木および各収納位置を対象に実験を行う.

\subsection{実験手順}
実験手順は以下の手順で行う.

\begin{enumerate}
  \item[(1)] 卓上に積み木を1個設置し,卓上の目盛りから各物体の$0^{\circ}xy$座標を直接計測し,記録する.
  \item[(2)] Pythonの開発環境Spyderを起動し,画像処理プログラムから空間フィルタリングをOFFにする.
  \item[(3)] 画像処理プログラムを実行後,マニピュレータの制御用ソフトウェアを用いてマニピュレータを手動操作し,画像に積み木が現れる位置まで手先を移動する.この際,手先の位置姿勢を記録する.
  \item[(4)] ヒストグラムを参考にしながら積み木のHSV色空間のしきい値(最大値,最小値)を設定し,記録する.また,スクリーンショット等を用いて,画像処理結果を記録する.
  \item[(5)] 得られた3次元位置情報を記録する.
  \item[(6)] 「平均化フィルタ」,「ガウシアンフィルタ」,「メディアンフィルタ」,「双方向フィルタ」において,それぞれ(1)~(5)を繰り返す.
  \item[(6)] 残りの積み木および収納位置を適所について,(1)~(6)を繰り返す.
  \item[(7)] 式(3.1)を用いて,座標変換を行い,物体の3次元位置をグローバル座標系で表現するエクセルファイルを作成し,計算する.
\end{enumerate}

\subsection{実験結果}
表6.1~表6.3に実験結果を示す.図6.1には,実位置と計測位置の誤差をまとめた図を示す.

\begin{table}[h]
  \centering
  \caption{積み木(赤)の結果}
  \begin{tabular}{|c|c|c|c|c|c|}
    \hline
                                                      & フィルタ無           & 平均フィルタ        & ガウシアンF         & メディアンF         & 双方向フィルタ      \\ \hline
    \hline
    実位置 ($x_{0}^{\circ}, y_{0}^{\circ}$) [mm]      & 488.5 ,-130.0        & 0                   & 0                   & 0                   & 0                   \\ \hline
    手先位置 ($x_{e}^{\circ}, y_{e}^{\circ}, z$) [mm] & 401.1 , 0.0 , 418.2  & 0                   & 0                   & 0                   & 0                   \\ \hline
    手先姿勢 (R, P, Y) [deg]                          & 180, 0 , 0           & 0                   & 0                   & 0                   & 0                   \\ \hline
    H(max, min)                                       & (3,0)                & (10,0)              & (10,0)              & (10,0)              & (10,0)              \\ \hline
    S(max, min)                                       & (255,60)             & (232,61)            & (255,60)            & (255,60)            & (255,60)            \\ \hline
    V(max, min)                                       & (255,0)              & (255,0)             & (255,0)             & (255,0)             & (255,95)            \\ \hline
    計測位置 ($x_{e}^{\circ}, y_{e}^{\circ}, z$) [mm] & (85.1 , 120.1 , 371) & (84.5,130.3,379.9)  & (85.0,128.6,377.9)  & (84.3,124.3,375.9)  & (85.0,126.9,375.9)  \\ \hline
    計測位置 ($x_{0}^{\circ}, y_{0}^{\circ}, z$) [mm] & (486.2,-120.1,47.2)  & (485.6,-130.3,38.3) & (486.1,-128.6,40.3) & (485.4,-204.3,42.3) & (486.1,-126.9,42.3) \\ \hline
  \end{tabular}
\end{table}

\begin{table}[h]
  \centering
  \caption{積み木(青)の結果}
  \begin{tabular}{|c|c|c|c|c|c|}
    \hline
                                                      & フィルタ無          & 平均フィルタ        & ガウシアンF         & メディアンF         & 双方向フィルタ      \\ \hline
    \hline
    実位置 ($x_{0}^{\circ}, y_{0}^{\circ}$) [mm]      & 588.5 ,-130.0       & 0                   & 0                   & 0                   & 0                   \\ \hline
    手先位置 ($x_{e}^{\circ}, y_{e}^{\circ}, z$) [mm] & 401.1 , 0.0 , 418.2 & 0                   & 0                   & 0                   & 0                   \\ \hline
    手先姿勢 (R, P, Y) [deg]                          & 180, 0 , 0          & 0                   & 0                   & 0                   & 0                   \\ \hline
    H(max, min)                                       & (160,140)           & (160,140)           & (160,140)           & (160,140)           & (160,140)           \\ \hline
    S(max, min)                                       & (255,60)            & (255,60)            & (255,60)            & (255,60)            & (255,60)            \\ \hline
    V(max, min)                                       & (255,0)             & (255,0)             & (255,0)             & (255,0)             & (255,0)             \\ \hline
    計測位置 ($x_{e}^{\circ}, y_{e}^{\circ}, z$) [mm] & (181.6,128.1,374.9) & (182.7,131.5,377.9) & (182.1,129.7,375.9) & (182.4,129.0,375.9) & (180.7,127.4,375.9) \\ \hline
    計測位置 ($x_{0}^{\circ}, y_{0}^{\circ}, z$) [mm] & (582.7,-128.1,43.3) & (583.1,-131.5,40.3) & (583.2,-129.7,42.3) & (583.5,-129.0,42.3) & (581.8,-127.4,42.3) \\ \hline
  \end{tabular}
\end{table}

\begin{table}[h]
  \centering
  \caption{積み木(黄)の結果}
  \begin{tabular}{|c|c|c|c|c|c|}
    \hline
                                                      & フィルタ無          & 平均フィルタ        & ガウシアンF         & メディアンF         & 双方向フィルタ      \\ \hline
    \hline
    実位置 ($x_{0}^{\circ}, y_{0}^{\circ}$) [mm]      & 388.5 ,-130.0       & 0                   & 0                   & 0                   & 0                   \\ \hline
    手先位置 ($x_{e}^{\circ}, y_{e}^{\circ}, z$) [mm] & 401.1 , 0.0 , 418.2 & 0                   & 0                   & 0                   & 0                   \\ \hline
    手先姿勢 (R, P, Y) [deg]                          & 180, 0 , 0          & 0                   & 0                   & 0                   & 0                   \\ \hline
    H(max, min)                                       & (50,30)             & (50,30)             & (50,30)             & (50,30)             & (50,30)             \\ \hline 
    S(max, min)                                       & (255,60)            & (255,60)            & (255,60)            & (255,60)            & (255,60)            \\ \hline
    V(max, min)                                       & (255,0)             & (255,0)             & (255,0)             & (255,0)             & (255,0)             \\ \hline
    計測位置 ($x_{e}^{\circ}, y_{e}^{\circ}, z$) [mm] & (-12.3,119.9,365.9) & (-15.0,124.7,370.9) & (-13.0,121.9,366.9) & (-13.0,117.2,365.9) & (-13.0,121.2,365.9) \\ \hline
    計測位置 ($x_{0}^{\circ}, y_{0}^{\circ}, z$) [mm] & (338.8,-119.9,52.3) & (386.1,-124.7,47.3) & (388.1,-121.9,51.3) & (388.1,-117.2,52.3) & (-13.0,121.2,365.9) \\ \hline
  \end{tabular}
\end{table}
\newpage

次に,図6.2に赤ブロックのフィルタ後の画像を示す.

\subsection{考察}
実位置と計測結果の誤差について,考察を行う.

\section{実験2-2:3次元位置計測と物体の把持収納実験}

\subsection{実験概要}
本実験では,手先に装着したRGB-Dカメラから把持物体および収納位置の3次元位置情報を取得し,
物体の把持および収納する実験を行う.図7.1に実験環境の模式図を示す.マニピュレータの卓上には,
把持する五角形の積み木(赤色,青色,黄色)と収納位置用の積み木(黄,青,緑)が置かれている.
マニピュレータは五角形形の積み木を検出・把持して収納位置に配置する動作を行う.

\subsection{実験手順}
実験手順は以下の手順で行う.

\begin{enumerate}
  \item[(1)] 卓上に積み木を3個、収納台座をすべて無造作に配置する.
  \item[(2)] Pythonの開発環境Spyderを起動し,実験3において最も優れた結果となった空間フィルタリングを選択する.
  \item[(3)] 画像処理プログラムを実行後,マニピュレータの制御用ソフトウェアを用いてマニピュレータを手動操作し,画像に各物体が現れる位置まで手先を移動する.この際,手先の位置姿勢を記録する.
  \item[(4)] 実験3で記録したHSV色空間のしきい値を設定し,各物体の検出を行う.また,実験3で作成したエクセルを用いて,各物体の3次元位置をグローバル座標系で表現する.
  \item[(5)] マニピュレータの制御用ソフトウェアを用いて,手先位置の数値指定によるオンラインティーチングを行う.動作は,赤積み木の把持収納,青積み木の把持収納,黄積み木の把持収納をすべて連続で行う.ただし,手先の姿勢$(R, P, Y) = (180, 0, 0)$ [deg]とし,手先の$0^{\circ}$位置を80[mm](把持時),110[mm](収納時)とすること.また,動作の様子を動画撮影する.
\end{enumerate}

\subsection{実験結果}
表7.1に実験結果を示す.また,図7.2にマニピュレータが積み木を把持・収納する様子を示す.

\begin{table}[h]
  \centering
  \caption{物体の位置}
  \begin{tabular}{|c|c|c|c|c|c|c|}
    \hline
                                                      & 赤積木 & 青積木 & 黄積木 & 赤収納              & 青収納              & 黄収納              \\ \hline
    \hline
    手先位置 ($x_{0}^{\circ}, y_{0}^{\circ}, z$) [mm] & 0      & 0      & 0      & 0                   & 0                   & 0                   \\ \hline
    手先姿勢 (R, P, Y) [deg]                          & 0      & 0      & 0      & 0                   & 0                   & 0                   \\ \hline
    計測位置 ($x_{e}^{\circ}, y_{e}^{\circ}, z$) [mm] & 0      & 0      & 0      & (177.7,126.6,375.9) & (88.7,-129.5,372.9) & (-5.7,-128.8,365.9) \\ \hline
    計測位置 ($x_{0}^{\circ}, y_{0}^{\circ}, z$) [mm] & 0      & 0      & 0      & (578.8,-126.6,42.3) & (489.9,129.545.3)   & (395.4,128.852.3)   \\ \hline
    成功                                              & 成功   & 成功   & 成功   & 成功                & 成功                & 失敗                \\ \hline 
    
  \end{tabular}
\end{table}

\subsection{考察}
次に,考察を行う.


%%%%%%%%%%%%%%%%%%%%%%%%%%%%%%%%%%%%%%%%%%%%%%%%%%%%%%%%%%%%%%%
\section{課題}

\subsection{逆運動学の導出}
課題:式(2.6),式(2.12)~式(2.14)を導出せよ.手書き可.
ただし,枠や枠内の文章は報告書に記載不要とする.参考文献は報告書最後の参考文献の欄に記載すること.

\subsection{回転行列とオイラー角の変換式}
課題:式(2.17)~式(2.20)を導出せよ.手書き可.ただし,枠や枠内の文章は報告書に記載不要とする.
参考文献は報告書最後の参考文献の欄に記載すること.

\subsection{空間フィルタリング}
「平均化フィルタ」,「ガウシアンフィルタ」,「メディアンフィルタ」,
「双方向フィルタ」の特徴についてそれぞれ100〜200字程度で説明せよ.
ただし,枠や枠内の文章は報告書に記載不要とする.参考文献は報告書最後の参考文献の欄に記載すること.

\subsection{色空間}
RGB色空間とHSV色空間について,それぞれ100〜200字程度で説明せよ.ただし,
枠や枠内の文章は報告書に記載不要とする.参考文献は報告書最後の参考文献の欄に記載すること.

\subsection{ロボットビジョンの実用例}
ステレオカメラ(単眼カメラでもよい)とロボットマニピュレータを統合することで
可能となる作業の実用例を1例以上調査し,それぞれ200字程度で説明せよ.
ただし,枠や枠内の文章は報告書に記載不要とする.参考文献は報告書最後の参考文献の欄に記載すること.

