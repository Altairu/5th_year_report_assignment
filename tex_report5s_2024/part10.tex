% !TEX root = main.tex

%%%%%%%%%%%%%%%%%%%%%%%%%%%%%%%%%%%%%%%%%%%%%%%%%%%%%%
\section{数式}
%%%%%%%%%%%%%%%%%%%%%%%%%%%%%%%%%%%%%%%%%%%%%%%%%%%%%%
\subsection{数式の使い方}
%%%%%%%%%%%%%%%%%%%%%%%%%%%%%%%%%%%%%%%%%%%%%%%%%%%%%%
\subsubsection{数式の使い方その 1}
%%%%%%%%%%%%%%%%%%%%%%%%%%%%%%%%%%%%%%%%%%%%%%%%%%%%%%
数式を文章中に入れるには,「{\verb+\( 式 \)+}」または「{\verb+$ 式 $+}」で囲んでああああああああああああああああああああああああああああああああああああああああああああああ
\(
\dot{x} = {A}{x} + {B}{u}
\) または $
  \dot{x} = Ax + Bu
$ のようにします.また,「{\verb+\[ 式 \]+}」で囲んで
\[
  \dot{x} = Ax + Bu
\]
とすれば,改行後に数式が挿入されます.式番号は「{\verb+\begin{equation} 式 \end{equation}+}」で囲んで
\begin{equation}
  \dot{x} = Ax + Bu
  \label{eq:state}
\end{equation}
とすれば自動的につきますし,(\ref{eq:state}) 式のように,式番号を参照することもできます.

%%%%%%%%%%%%%%%%%%%%%%%%%%%%%%%%%%%%%%%%%%%%%%%%%%%%%%
\subsubsection{数式の使い方その 2}
%%%%%%%%%%%%%%%%%%%%%%%%%%%%%%%%%%%%%%%%%%%%%%%%%%%%%%
「{\verb+\begin{align} 式 \end{align}+}」で囲んで
\begin{align}
  E\dot{x} & = Ax + Bu \\
  y        & = Cx + Du
\end{align}
とすれば,「{\verb+& そろえたい部分+}」で囲まれた位置 (``\,=\,'') をそろえることができます.{\verb+\nonumber+} で
\begin{align}
  E\dot{x} & = Ax + Bu           \\
  y        & = Cx + Du \nonumber
\end{align}
のように式番号をはずすこともできます.

以前は「{\verb+\begin{eqnarray} 式 \end{eqnarray}+}」で囲んで
\begin{eqnarray}
  E\dot{x} &=& Ax + Bu \\
  y &=& Cx + Du
\end{eqnarray}
とし,「{\verb+& そろえたい部分 &+}」としていました.現在では,推奨されない書き方です.

%%%%%%%%%%%%%%%%%%%%%%%%%%%%%%%%%%%%%%%%%%%%%%%%%%%%%%
\subsection{ボールドイタリック}
%%%%%%%%%%%%%%%%%%%%%%%%%%%%%%%%%%%%%%%%%%%%%%%%%%%%%%
「{\verb+{\bm ボールドイタリックにしたい部分}+}」により数式中の文字をボールドイタリックにすることができます.
\begin{equation}
  \dot{\bm x} = {\bm A}{\bm x} + {\bm B}{\bm u}
\end{equation}
ああああああああああああああああああああああああああ
ああああああああああああああああああああああああああ
ああああああああああああああああああああああああああ
ああああああああああああああああああああああああああ
ああああああああああああああああああああああああああ
ああああああああああああああああああああああああああ
ああああああああああああああああああああああああああ
ああああああああああああああああああああああああああ
ああああああああああああああああああああああああああ
ああああああああああああああああああああああああああ
ああああああああああああああああああああああああああ
ああああああああああああああああああああああああああ

%%%%%%%%%%%%%%%%%%%%%%%%%%%%%%%%%%%%%%%%%%%%%%%%%%%%%%
\subsection{ディスプレイ形式の分数}
%%%%%%%%%%%%%%%%%%%%%%%%%%%%%%%%%%%%%%%%%%%%%%%%%%%%%%
「{\verb+\frac{分子}{分母}+}」で表現された分数を行列中などに用いると,
\begin{equation}
  \left[ \begin{array}{cc}
      a               & \frac{c}{b} \\
      \frac{e}{b + c} & f
    \end{array} \right]
\end{equation}
のように分数が小さくなってしまいます.通常の大きさにしたい場合には,「{\verb+\frac{分子}{分母}+}」の代わりに「{\verb+\dfrac{分子}{分母}+}」を使用して下さい.「{\verb+\dfrac{分子}{分母}+}」を用いると,次式のようになります.
\begin{equation}
  \left[ \begin{array}{cc}
      a                & \dfrac{c}{b} \\
      \dfrac{e}{b + c} & f
    \end{array} \right]
\end{equation}
